\documentclass{beamer}

\usepackage[utf8]{inputenc} % UTF-8
\usepackage[T1]{fontenc} % UTF-8
\usepackage{lmodern} % fonts
\usepackage{pdfcomment} % pdf notes
\usepackage{multicol} % multicols env

%% Searpkernotes %%
\newcommand{\speakernote}[1]{\pdfmargincomment[icon=note]{#1}}

%% Theme varsaw %%
\usetheme{Warsaw}

%% 2 columns in beamer header %%
\makeatletter
\def\insertsectionnavigation#1{\hbox to #1{\vbox{{\usebeamerfont{section in head/foot}\usebeamercolor[fg]{section in head/foot}\def\slideentry##1##2##3##4##5##6{}\def\sectionentry##1##2##3##4##5{\ifnum##5=\c@part\def\insertsectionhead{##2\hskip1em}\def\insertsectionheadnumber{##1}\def\insertpartheadnumber{##5}\hyperlink{Navigation##3}{\ifnum\c@section=##1{\usebeamertemplate{section in head/foot}}\else{\usebeamertemplate{section in head/foot shaded}}\fi}\par\fi}\parbox[c][0cm][c]{.5\paperwidth}{\begin{multicols}{2}\dohead\end{multicols}}\space}}\hfil}}
\def\insertsubsectionnavigation#1{\hbox to #1{\vbox{{\usebeamerfont{subsection in head/foot}\usebeamercolor[fg]{subsection in head/foot}\vskip0.5625ex\beamer@currentsubsection=0\def\sectionentry##1##2##3##4##5{}\def\slideentry##1##2##3##4##5##6{\ifnum##6=\c@part\ifnum##1=\c@section\ifnum##2>\beamer@currentsubsection\beamer@currentsubsection=##2\def\insertsubsectionhead{##5}\def\insertsectionheadnumber{##1}\def\insertsubsectionheadnumber{##2}\def\insertpartheadnumber{##6}\beamer@link(##4){\ifnum\c@subsection=##2{\usebeamertemplate{subsection in head/foot}}\else{\usebeamertemplate{subsection in head/foot shaded}}\fi\hfill}\par\fi\fi\fi}\hspace*{0.5em}\parbox[c][0cm][c]{\dimexpr.5\paperwidth-1em\relax}{\begin{multicols}{2}\dohead\vskip0.5625ex\end{multicols}}\space}\hfil}}}
\setbeamertemplate{headline}{\leavevmode\@tempdimb=2.4375ex\ifnum\beamer@subsectionmax<\beamer@sectionmax\multiply\@tempdimb by\beamer@sectionmax\else\multiply\@tempdimb by\beamer@subsectionmax\fi\ifdim\@tempdimb>0pt\advance\@tempdimb by 1.125ex\begin{beamercolorbox}[wd=.5\paperwidth,ht=0.5\@tempdimb,dp=2ex]{section in head/foot}\vbox to0.5\@tempdimb{\vfill\insertsectionnavigation{.5\paperwidth}\vfill}\end{beamercolorbox}\begin{beamercolorbox}[wd=.5\paperwidth,ht=0.5\@tempdimb,dp=2ex]{subsection in head/foot}\vbox to0.5\@tempdimb{\vfill\insertsubsectionnavigation{.5\paperwidth}\vfill}\end{beamercolorbox}\fi}
\makeatother

%% Add footline %%
\beamertemplatenavigationsymbolsempty
\setbeamerfont{page number in head/foot}{size=\large}
\setbeamertemplate{footline}[frame number]
\logo{\includegraphics[height=15mm]{medias/logo.png}}

\title[CSE-ProdCons]{CSE\\Problème des producteurs et consommateurs}
\subtitle{}
\author{Quentin DUNAND, Rémi GATTAZ}
\date{17 Décembre 2015}
\subject{producteur,consommateur,threads,semaphore,concurrent}


\begin{document}

    % Title frame
    \frame{
        \titlepage
    }

    % Table of Contents
    \begin{frame}
        \frametitle{Summary}
        \setcounter{tocdepth}{1}
        \tableofcontents
    \end{frame}

    \section{Le problème}

    \begin{frame}
        \frametitle{Le problème des producteurs et des consommateurs}

        \begin{itemize}
            \item Problème classique de multithreading
            \item[ ]
            \item Production de messages
            \item[ ]
            \item Consommation de mesages
        \end{itemize}
    \end{frame}

    \begin{frame}
        \frametitle{Les conditions d'acceptations}

        \begin{itemize}
            \item[1] Buffer utilisé de façon optimal. Maximum de sa capacité ou de la capacité des producteurs
            \item[2] Terminaison du programme : production de messages terminés et tous les messages consommés.
            \item[3] Message déposé à la date T consommé avant message déposé à la date T+n.
            \item[4] Consommateur se présentant à la date T consomme avant un consommateur se présentant à la date T+n.
        \end{itemize}
    \end{frame}

    \section{Les objectifs}

    \subsection{Objectif 1}
    \begin{frame}
        \frametitle{Objectif 1}
        \begin{block}{Nos choix}
            Implémentation naïve du problème.\\
            Utilisation des méthodes wait() et notifyAll()
        \end{block}

        \begin{alertblock}{Problème}
            Vol de cycles
        \end{alertblock}
    \end{frame}


    \subsection{Objectif 2}
    \begin{frame}
        \frametitle{Objectif 2}
        \begin{block}{Nos choix}
            Création d'une classe Semaphore
        \end{block}

        \begin{exampleblock}{Le gain}
            Plus de vol de cycles
        \end{exampleblock}

        \begin{alertblock}{Problème}
            Une sémaphore parfois non bloquante
        \end{alertblock}
    \end{frame}


    \subsection{Objectif 3}
    \begin{frame}
        \frametitle{Objectif 3}
        \begin{block}{Nos choix}
            Ajout de l'Observateur
        \end{block}

        \begin{exampleblock}{Le gain}
            Apporte la vérification de certaines conditions d'acceptations
        \end{exampleblock}
    \end{frame}



    \subsection{Objectif 4}
    \begin{frame}
        \frametitle{Objectif 4}
        Ajout de deux conditions :
        \begin{itemize}
            \item Blocage des producteurs avant n consommations
            \item Message retiré après n consommations
        \end{itemize}

        \begin{alertblock}{Problème}
            Imbrication de blocs synchronisés
        \end{alertblock}

        \begin{exampleblock}{Solution}
            Ajout de 2 Maps dans ProdCons
            \begin{itemize}
                \item Map contenant des Sémaphores
                \item Map contenant les nombres d'exemplaires
            \end{itemize}
        \end{exampleblock}
    \end{frame}

    \begin{frame}
        \frametitle{Objectif 4}

        \begin{alertblock}{Problème}
            Information nombre d'exemplaires
        \end{alertblock}

        \begin{exampleblock}{Solution}
            Cast de Message en MessageX
        \end{exampleblock}
    \end{frame}

    \subsection{Objectif 5}
    \begin{frame}
        \frametitle{Objectif 5}
        \begin{block}{Nos choix}
            Utilisation de Lock, Signal et Condition
        \end{block}

        \begin{exampleblock}{Solution}
            Une alternative à Sémaphore
        \end{exampleblock}
    \end{frame}

    \subsection{Objectif 6}
    \begin{frame}
        \frametitle{Objectif 6}
        \begin{block}{Nos choix}
            Utilisation d'une Queue\\
            Utilisation de deux HashSet\\
            Utilisation de deux HashTable
        \end{block}

        \begin{alertblock}{Problème}
            Mauvais compréhension du sujet
        \end{alertblock}

        \begin{exampleblock}{Solution}
            Vérification du FIFO dans le buffer des Messages
        \end{exampleblock}
    \end{frame}

    \section{Conclusion}
    \begin{frame}

        \begin{alertblock}{Problème}
            Problème de la sémaphore identifié mais pas de solution trouvé.
        \end{alertblock}

        \begin{exampleblock}{Améliorations envisagées}
            Utilisation de Observateur.coherent()\\
            Gestion des exceptions dans les logs\\
        \end{exampleblock}

    \end{frame}

    \begin{frame}

        Merci pour votre attention

    \end{frame}

\end{document}
